\chapter{Conclusioni}

Abbiamo sperimentato con un approccio puramente appreso per il problema dell'approximate set membership, con l'obiettivo di giungere a un filtro di Bloom composto interamente da modelli di machine learning, come possibile alternativa più efficiente rispetto ai filtri di Bloom classici e a quelli parzialmente appresi. In particolare, abbiamo sviluppato un filtro configurato come una sequenza di classificatori. In una prima variante, tale sequenza è composta da support vector machine lineari, seguite da un albero di decisione finale; nella seconda variante, invece, abbiamo reti neurali, anch'esse seguite da un albero di decisione. 

Dopo aver misurato su alcuni insiemi di dati le prestazioni di diversi filtri puramente appresi, abbiamo osservato che, in genere, questo approccio conviene rispetto a quello classico, non appreso, ma non conviene rispetto a quello parzialmente appreso. 
In ogni esperimento condotto, utilizzare una catena composta interamente da modelli di machine learning non è stato vantaggioso rispetto a una avente al suo termine un filtro di Bloom classico. Inoltre, all'aumentare della lunghezza della sequenza, abbiamo riscontrato un peggioramento dell'efficienza in termini di spazio occupato. 
Tuttavia, anche se la sequenza di modelli ottimale non è puramente appresa, siamo giunti a un approccio innovativo per aggiungere, controllare e dimensionare i singoli classificatori che la compongono. Ulteriori sperimentazioni, ad esempio sfruttando altre tipologie di modello, potrebbero portare a nuovi interessanti risultati. 