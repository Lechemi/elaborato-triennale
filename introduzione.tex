\chapter{Introduzione}

In uno scenario, come quello attuale, in cui la quantità di dati prodotti, accumulati e scambiati è ai massimi storici, disporre di strutture dati efficienti sia dal punto di vista dello spazio di memoria occupato, sia da quello del tempo di accesso è di importanza fondamentale. 
Il machine learning (ML), che negli ultimi tempi ha assunto un ruolo centrale in numerosi ambiti applicativi, si sta rivelando un potente strumento anche in questo campo. È infatti emerso il ramo delle strutture dati apprese, che adottano modelli di ML in congiunzione con strutture dati tradizionali per catturare e sfruttare i pattern spesso presenti nei dati, migliorando l'efficienza di queste ultime.

Il presente lavoro si è concentrato su un particolare tipo di struttura dati: il filtro di Bloom (Bloom Filter, BF). Il BF è ampiamente impiegato per affrontare il problema dell'Approximate Set Membership (ASM), che consiste nella verifica dell'appartenenza di un elemento a un insieme, tollerando un certo tasso di falsi positivi a favore di un aumento dell'efficienza spaziale e temporale. Trattandosi di un problema carico di risvolti pratici, già esistono delle varianti apprese di questa struttura dati, che combinano uno o più modelli di ML con un BF classico. Tuttavia, l'obiettivo di questo studio è stato quello di esplorare la nuova possibilità di un filtro puramente appreso (Fully Learned Filter, FLF), basato esclusivamente su modelli di ML, senza l'ausilio di un filtro di Bloom tradizionale.

In primo luogo, è stato sviluppato l'aspetto teorico del FLF, giungendo a una modellazione matematica del filtro che ha costituito le fondamenta per la parte di implementazione, svolta nel linguaggio Python. Quest'ultima ha attraversato diverse iterazioni prima di raggiungere la versione attuale. Durante l'intero processo di realizzazione del FLF, sono stati condotti esperimenti su diversi insiemi di dati, al fine di valutare le prestazioni del filtro stesso, e nel caso, apportare modifiche all’implementazione.

Questo lavoro mi ha permesso di acquisire e approfondire i concetti fondamentali del ML, tra cui la definizione e il funzionamento di un modello, le tecniche di addestramento, la valutazione delle prestazioni e la conduzione di esperimenti, anche toccando concetti come parallelizzazione e serializzazione. Inoltre, mi ha offerto l'opportunità di esplorare alcune tipologie di modelli di ML, quali Support Vector Machines (SVM) lineari, reti neurali (che, al momento, sono al centro dell'attenzione in ambito ML) e alberi di decisione. Parallelamente, è stato necessario svolgere uno studio sui filtri di Bloom e sulle loro varianti apprese.
Dal punto di vista pratico, ho avuto l'occasione di utilizzare diverse librerie Python utili in campo ML; prima tra tutte, \textit{scikit-learn}, che offre una vasta serie di classi e funzioni per la manipolazione dei dati, la costruzione, l’addestramento e la valutazione di modelli di ML.

I prossimi capitoli sono organizzati come segue: il Capitolo 2 introduce il problema dell'approximate set membership e i filtri di Bloom; il Capitolo 3 offre una panoramica sul machine learning, descrivendo i concetti fondamentali e tre tipologie di modelli rilevanti per questo elaborato; il Capitolo 4 approfondisce la trattazione teorica dei filtri puramente appresi, mentre il Capitolo 5 presenta i risultati pratici con le relative osservazioni. Seguono, infine, le conclusioni.